%%
% 引言或背景
% 引言是论文正文的开端,应包括毕业论文选题的背景、目的和意义;对国内外研究现状和相关领域中已有的研究成果的简要评述;介绍本项研究工作研究设想、研究方法或实验设计、理论依据或实验基础;涉及范围和预期结果等。要求言简意赅,注意不要与摘要雷同或成为摘要的注解。
% modifier: 黄俊杰(huangjj27, 349373001dc@gmail.com)
% update date: 2017-04-15
%%

\chapter{引言}
%定义,过去的研究和现在的研究,意义,与图像分割的不同,going deeper
\label{cha:introduction}
\section{选题背景与意义}
\label{sec:background}
% What is the problem
% why is it interesting and important
% Why is it hards, why do naive approaches fails
% why hasn't it been solved before
% what are the key components of my approach and results, also include any specific limitations,do not repeat the abstract
%contribution

通用斑马线检测,意在提出对于分辨率,拍摄角度,光照条件,画面占比没有特殊苛刻要求的斑马线检测和识别算法。问题的关键有两部分,一是如何精准检出斑马线,二是如何提高系统的鲁棒性,降低识别系统对于附加条件的要求。由于自动驾驶技术的兴起和导盲系统的完善,对于交通标志识别的需求日益迫切。斑马线是行车路口和人车交汇点的重要标志,在自动驾驶的行车区域划定和盲人的行走辅助当中都有重要作用。目前与交通标志识别有关的领域分别使用了一些领域相关的斑马线识别技术:例如遥感领域、自动驾驶领域、盲人辅助领域,各种技术都有各自的长处和局限,通过提供一个通用的高准确率低误识率的斑马线的检测与定位技术,可以为这些领域提供极大的便利。斑马线的视觉特征非常明显,但是其检测算法并不直观,简单的算法往往会陷入窘境,例如道路标志褪色造成的“白线”变“灰线”,或者会被一只黑白相间的猫所迷惑。本题提出的算法,在基本保证实时性的同时,能够利用多种判据规避相似图形,同时对于视角等条件没有要求。


\section{国内外研究现状和相关工作}
\label{sec:related_work}
京都工业大学的Mohammad Shorif Uddin和Tadayoshi Shioyama在2005年提出将双极系数法\cite{bipolar}应用于盲人辅助行走系统中识别斑马线,双极系数法运算量小结果比较准确受到广泛传播,二人之后再次提出基于IPM变换的改进方法。2013年北方工业大学的王一丁和徐超改良了检测流程中的IPM变换\cite{ipm}。在2015年,武汉大学测绘学院的阎利和黄亮将双极系数法应用于高空球形摄像机中,并且结合了Hough变换来提取直线。为了黄新和林倩,在2017年进一步提出了双极系数和直线提取的一些改进方法。斑马线识别实际是物体识别的一个特例,因此本文也涉及物体识别和图像分割的通用方法,例如Pedro F. Felzenszwalb和Daniel P. Huttenlocher提出的基于图论的图像分割算法\cite{segmentation},被广泛应用来实现Selective Search,产生深远影响。

\section{本文的论文结构与章节安排}

\label{sec:arrangement}
本文共分为五章,各章节内容安排如下:

第一章引言。

第二章图像特征提取,本章介绍传统图像处理方法在斑马线识别这个具体问题上的应用技术,比较多种方法的优势和劣势,详细介绍和讨论双极系数法和灰度共生矩阵及其特征值。

第三章兴趣区域生成,本章重点介绍一种高效的图像分割算法,以及图像分割在斑马线识别问题中减少计算量和提高准确性等重要意义。

第四章检测与识别算法,本章从双极系数、Hough变换到纹理提取,完整的串联出斑马线检测的程序,并给出详细的中间过程说明。

第五章试验结果,本章包括本文介绍的多种方法在测试集上的准确率和误识率数据,和典型已知错误的分析。

第六章总结与展望,是本文的最后一章,是对本文内容的整体性总结以及对未来工作的展望。

