\chapter{图像特征提取}
\label{cha:feature}

\section{双极系数}
\label{sec:bipolar}
双极系数法最早由京都工业大学的Mohammad Shorif Uddin和Tadayoshi Shioyama提出,作为一种鲁棒的斑马线检测算法,替代实时性和准确性不足的“消逝点法”。它的鲁棒性表现在不受多种光照条件和气候变化的影响。但该特征受道路磨损的影响较大。 \par
它的算法描述如下:对于一副图像,将它的灰度分布记为$p_0(x)$。将灰度像素分为低灰度区域和高灰度区域,那么$p_0(x)$可以表示为
\begin{align}
	p_0(x)=\alpha p_1(x) + (1-\alpha)p_2(x)
\end{align}
其中$p_1(x)$表示低灰度的分布,$p_2(x)$表示高灰度的分布。以上变量的均值和方差可以表示为
\begin{align}
	\mu_i=\int_{-\infty}^{+\infty}xp_i(x)dx, (i = 1, 2, 3) \\
	\sigma_i^2=\int_{-\infty}^{+\infty}(x-\mu_i)^2p_i(x)dx, (i = 1, 2, 3)
\end{align}
综合以上三式,
\begin{align}
	\sigma_0^2=\alpha \sigma_1^2+(1-\alpha)\sigma_2^2+\alpha(1-\alpha)(\mu_1-\mu_2)^2
\end{align}
观察(2.4)可以发现,灰度分布的总体方差可以表示为两部分方差和均值差平方的加权和,那么当总体方差主要受均值差影响时,我们认为图像是完全双极性的,既$\sigma_0^2\approx \alpha(1-\alpha)(\mu_1-\mu_2)^2$。所以定义双极性的函数为:
\begin{align}
	\gamma \equiv \frac{1}{\sigma_0^2} \alpha(1-\alpha)(\mu_1-\mu_2)^2 
\end{align}\par
式(2.5)告诉我们,$0 \leq \gamma \leq 1$,并且当$\gamma=1$时,说明低灰度分布和高灰度分布的内部方差均为0,此时图像具有完全的双极性。当$\gamma=0$时,低灰度分布和高灰度分布均值相同或某一个灰度区域不存在,既图像完全没有双极性。

\section{灰度共生矩阵和特征值}
\label{sec:cogrey}
斑马线有黑白相间,间隔相同的特线,因此可以看做一种纹理,通过灰度共生矩阵(GLCM,Gray-Level Co-occurrence Matrix)将纹理信息提取出来。纹理是由灰度分布在空间域上反复出现而形成的,因此图像中相隔某个角度某个距离的像素的灰度之间有相关性。共生矩阵是指给定偏移量后共现灰度值的分布。它有如下特点:
\begin{enumerate}
	\item 偏移量是指,$(\Delta x, \Delta y)$或者$(\Delta \rho, \Delta \theta)$,$x$和$y$分别表示偏移的横纵坐标,$\rho$和$\theta$分别表示偏移的圆心距和极角
	\item 一个灰度定义域为$[0,p]$的图像的灰度共生矩阵是一个$p \times p$的矩阵
	\item 灰度共生矩阵中坐标为$(x, y)$的位置的值是$x$和$y$这两个灰度在给定偏移量下共同出现的次数
\end{enumerate}\par
对于一个矩阵A来说,它的灰度共生矩阵C可以表示为
\begin{align}
C_{\Delta x, \Delta y}(i,j)=\sum_{x=1}^n\sum_{y=1}^m\begin{cases} 1, & \text{if }A_{i,j}\text{ and }A_{x+\Delta x, y+\Delta y}=j \\ 0, & \text{otherwise}\end{cases}, (offset \equiv (\Delta x, \Delta y))
\end{align}
然而求出的灰度共生矩阵维数很大,无法很直观的使用,因此介绍与本题相关的灰度共生矩阵的特征值:熵(entropy),能量(energy),相关性(correlation),对比度(contrast)。
\subsection{熵}
\begin{align}
ENT=\sum_{i,j\in A} -log(p_{i,j})p_{i,j}
\end{align}\par
把信息中排除了冗余后的平均信息量是信息熵,灰度共生矩阵的信息熵反映了图像邻域内的信息熵大小,越复杂的纹理所计算出的熵越大,反之较小。斑马线的特征是其灰度共生矩阵的熵较小。
\subsection{能量}
\begin{align}
ASM=\sum_{i,j\in A} p_{i,j}^2
\end{align}\par
能量是反映图像灰度分布均匀和纹理粗细的重要指标,当纹理较细致纹理问分布均匀时,灰度共生矩阵的能量较大,反之较小。斑马线体现出的特征是其灰度共生矩阵的能量不能过大也不能过小。可以观察到,灰度全为最低灰度时是灰度共生矩阵的能量的下界,而多彩的雪花图是灰度共生矩阵的能量的上界。能量也称角二阶矩和统一度。
\subsection{相关性}
\begin{align}
COR=\sum_{i, j\in A} p_{i,j} \frac{(i-\mu_i)(j-\mu_j)}{\sqrt{\sigma_i^2\sigma_j^2}}
\end{align}\par
相关性是像素和其邻域是否相关的度量,取值范围在$[-1, 1]$之间
\subsection{对比度}
\begin{align}
CON=\sum_{i, j\in A} p_{i,j}(i-j)^2
\end{align}\par
对比度是图像中黑与白的层次性,当纹理黑白分明,对比明显,视觉效果强烈时,其对应的灰度共生矩阵体现出高对比度的特征,反之,如果图像的纹理灰暗,亮处和暗处不易分辨,则体现出低对比度。
\subsection{反差分矩}
\begin{align}
IDM=\sum_{i, j\in A} \frac{p_{i,j}}{1+(i-j)^2}
\end{align}\par
反差分矩反映了纹理是否清晰和是否规则,纹理清晰、易于识别、富有规律、易于描述的其共生矩阵的反差分矩较大,反之较小。斑马线对应较大的反差分矩。反差分矩又称齐次性。