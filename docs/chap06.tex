%%
% 结论
% 结论是毕业论文的总结,是整篇论文的归宿,应精炼、准确、完整。结论应着重阐述自己的创造性成果及其在本研究领域中的意义、作用,还可进一步提出需要讨论的问题和建议。
% modifyer: 黄俊杰(huangjj27, 349373001dc@gmail.com)
% update date: 2017-04-13
%%
\chapter{总结与展望}
\section{工作总结}
本次工作由传统的双极系数法入手,探索双极系数法的优点和局限,顺藤摸瓜提出了空间域上的特征检测——Hough判据,与之前文章的不同之处在于,本文不单单考虑Hough变换后直线的个数和间隔,更考虑二值化后的颜色黑白间隔的特征进一步提高了识别的准确率。进一步创新地利用纹理检测的手段,从新的角度看待斑马线检测,并将该问题变成了一个可以应用机器学习的问题。
\section{研究展望}
首先一些如第五章中列出的已知错误,可以通过一些更加复杂和精确的技术来解决,这一部分有待填补。当抽象出纹理的特征向量后,可以有多种机器学习方法可以应用,例如决策树和神经网络,另外在训练数据和测试集方面还有增加的余地,对于三种判据优劣的具体分析还可以深入。斑马线检测,除了传统的特征检测与定位技术,是否可以看成通用的物体检测,这一点有待探究。本文的实验平台以python为主,包换OpenCV、skimage和sklearn,平均每张图片的处理需要4到5秒,这个速度如果用C++来重构代码,大约可以支撑8帧每秒的实时图片流,对于实时性要求基本满足,但提升空间还很大。